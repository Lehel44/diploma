\pagenumbering{roman}
\setcounter{page}{1}

\selecthungarian

%----------------------------------------------------------------------------
% Abstract in Hungarian
%----------------------------------------------------------------------------
\chapter*{Kivonat}\addcontentsline{toc}{chapter}{Kivonat}

A biometrikus azonosításra használható modern, neurális hálózat alapú beszélőfelismerés jelenleg széles körben kutatott, gyorsan fejlődő terület. Jelenleg nincs átfogó rendszerezés, összehasonlítás a rendszerek között és ingyenesen elérhető beszélőfelismerő alkalmazás sem különböző modellek teljesítményének összeméréshez.
\newline
\newline
Dolgozatomban bemutatom a korai beszélőfelismerő rendszereket, a mai modern neurális hálózat alapú megközelítéseket és az ingyenesen elérhető beszédadatbázisokat. Ismertetem a metatanítás fogalmát a hozzá kapcsolódó optimizációs technikákkal. Részletesen bemutatok két beszélőfelismerő modellt; egy módosított WaveNet architektúrát és egy sziámi konvolúciós hálózatot, amelyeken méréseket is végzek.
\newline
\newline
Végül leírom az általam készített Android alapú beszélőfelismerő alkalmazás implementációs részleteit, optimizálását és a felhasználási lehetőségeit. Az alkalmazást 8 beszélővel teszteltem és kiértékeltem az eredményeket, amelyek jól mutatják, hogy mesterséges környezetben tanított modellek nem viselkednek az elvárt módon valós körülmények között.


\vfill
\selectenglish


%----------------------------------------------------------------------------
% Abstract in English
%----------------------------------------------------------------------------
\chapter*{Abstract}\addcontentsline{toc}{chapter}{Abstract}

Modern neural network-based speech recognition for biometric identification is a widely researched and rapidly evolving field. There is currently no comprehensive systematisation, system comparison, and free speech recognition application for testing and to compare performance across different models.
\newline
\newline
In my thesis I introduce early speech recognition systems, today's modern neural network-based approaches and free speech recognition datasets. I introduce the concept of meta-learning with the related optimization techniques. I present in detail two models of speech recognition; a modified WaveNet architecture and a Siamese convolutional network, on which I also make measurements.
\newline
\newline
Finally, I describe the architecture, implementation details, optimization, and applications of my Android-based speech recognition application.
I tested the application with 8 speakers and evaluated the results, which show that models trained in artificial environments do not behave as expected in real-world conditions.


\vfill
\selectthesislanguage

\newcounter{romanPage}
\setcounter{romanPage}{\value{page}}
\stepcounter{romanPage}