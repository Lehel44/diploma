%----------------------------------------------------------------------------
\chapter{Háttérismeretek}

\section{Biometrikus azonosítás}

A biometria az emberek fizikai jellemzőinek mérésével és elemzésével foglalkozik. Alkalmazását tekintve három területet különböztetünk meg:

\begin{itemize}
	\item Felhasználó ellenőrzés: Az azonosító rendszer a biometrikus adatot egy, korábban vetthez hasonlítja. Ez alapján dönt, hogy a felhasználó hozzáférhet-e a kívánt erőforráshoz. Ilyen egy ujjlenyomat-olvasóval ellátott mobiltelefonon a képernyőzár feloldása. A felhasználó ellenőrzés arra ad választ, hogy az illető az-e akinek mondja magát.
	\item Felhasználó azonosítás: Az azonosító rendszer a biometrikus adatot több korábban vett mintához hasonlítja és arra ad választ, hogy ki a felhasználó; azaz beletartozik-e a korábban eltárolt biometrikus adatokból álló csoportba vagy nem. Ilyen lehet például egy ujjlenyomat-leolvasóval ellátott beléptetőrendszer cégek esetében.
	\item Duplikátum detektálás: Annak ellenőrzése, hogy egy felhasználó egynél többször szerepel-e egy adatbázisban. Csalások, például szociális támogatást többször igénylők kiszűrésére használják.
\end{itemize}

Az első biometrikus azonosítási eljárás az ujjlenyomatvételen alapuló személyiségazonosítás volt, amely a modern kriminalisztika világában terjedt el, de manapság már megtalálható okostelefonokban, biometrikus beléptetőrendszerekben is.

A biometrikus azonosítást az ún. biometrikus azonosító rendszer végzi el. A folyamat során a biometrikus azonosító rendszer mintát vesz az azonosítandó egyén egy vagy több előre meghatározott fizikai jellemzőjéről, és ezekről digitális lenyomatokat képez. Az első, regisztrációs fázisban a biometrikus minta lenyomatát a rendszer egy adatbázisban eltárolja, majd később az azonosítás során az aktuális mintát összeveti a korábban rögzítettel és dönt az egyezésről. Ahhoz, hogy az ember egy fizikai jellemzőjét biometrikus adatként használhassuk, a következő elvárásokat támasztjuk vele szemben:

\begin{itemize}
	\item Általánosság: A biometrikus adattal minden egyénnek rendelkeznie kell.
	\item Egyediség: A biometrikus adatnak egyedinek kell lennie a releváns populáción belül.
	\item Állandóság: A biometrikus adat nem, vagy csak keveset változzon az idő eltetével.
	\item Mérhetőség: Az biometrikus adat az egyén részéről legyen könnyen mérhető testi adottság.
	\item Teljesítmény: A biometrikus azonosító rendszerek teljesítménye: gyorsaság, pontosság, technológia.
	\item Elfogadottság: A releváns populáción belül a mérési eljárás mennyire elfogadott (emberi méltóság megőrzése).
	\item Biztonság: Mennyire nehéz utánozni, hamisítani a biometrikus adatot?
\end{itemize}

A biometrikus adat lehet fiziológiai (DNS, arc, ujjlenyomat, írisz) vagy viselkedési (hang, írás, gesztusok). Mivel ezek az adatok statisztikai jellegűek, megbízhatóságuk változó. Minél több adat van egy mintában, annál egyedibb, és minél nagyobb a releváns populáció (eltárolt minták összessége), annál valószínűbb, hogy találunk két hasonló mintát. Ennek elkerülésére manapság terjednek a multimódusú biometrikus azonosító rendszerek, amelyek több biometrikus adatot felhasználva végzik ez az ellenőrzés, azonosítás és duplikátum detektálás feladatát. 


\section{Beszélőfelismerés}

Az emberi kommunikáció során fontos feladat a beszélő partner felismerése. A telekommunikációs technológia fejlődése miatt elterjedt a telefonon vagy interneten történő hangalapú kommunikáció; a telefonos felhasználófelismerés mint biometrikus azonosítási módszer megjelent már banki alkalmazásokban, call centerekben és az elektronikus kereskedelemben is (mobiltelefonos vásárlás). Az elektronikus kommunikáció során sokszor csak a beszélő hangjára hagyatkozhatunk, az alapján ismerhetjük fel az illetőt. A beszélőfelismerést háromféle módon végezhetjük:

\begin{itemize}
	\item Naiv beszélőfelismerés: Az emberi, naiv beszélőfelismerés során az ismerős hangokat meglepően nagy pontossággal ismerjük fel.
	\item Törvényszéki beszélőfelismerés: A törvényszéki szakértői vizsgálat eredménye.
	\item Automatikus beszélőfelismerés: A beszélőfelismerést számítógépes rendszer végzi.
\end{itemize}

A beszélőfelismerés alatt három szűkebb fogalmat értünk. Ha a folyamat során az ismeretlen beszélőről azt ellenőrizzük, hogy az-e akinek állítja magát beszélő ellenőrzésről van szó. Beszélő szegmentáláskor a hangmintát homogén csoportokra bontjuk a beszélő személye alapján. Végül beszélő azonosításról beszélünk, ha az illető hangját rögzített hangok egy csoportjával vetjük össze és azt szeretnénk eldönteni, hogy melyikhez hasonlít a legjobban. Utóbbi felveti a kérdést, hogy mi történik ha a beszélő nem tagja a csoportnak. 

Emiatt megkülönböztetjük a nyitott és zárt halmazú beszélőazonosítást. Utóbbi esetén csak olyan beszélőket ismerünk fel, akikről van hangminta az adatbázisban, míg az előbbinél ismeretlen beszélők is megjelenhetnek, így ezt is kezelni kell.

A beszélőfelismerés továbbá lehet szöveg-függő és szöveg-független attól függően, hogy a felismerő rendszer egy előre meghatározott mondatot vár, vagy bármilyen hangminta alapján működik.

